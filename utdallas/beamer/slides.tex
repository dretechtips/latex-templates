\documentclass{beamer}

% Some common packages
\usepackage{graphicx, color}
\usepackage{alltt}
\usepackage{booktabs, calc, rotating}
\usepackage{natbib}
\usepackage{multicol}
\usepackage{amsmath, amsbsy, amssymb, amsthm, graphicx}
\usepackage[english]{babel}
\usepackage{xkeyval} 
\usepackage{xfrac}
\usepackage[normalem]{ulem}
\usepackage{fancyvrb} 
\usepackage{tikz, geometry, tkz-graph, xcolor}
\usepackage[latin1]{inputenc}
\usepackage{times}
\usepackage[T1]{fontenc}
\usepackage[export]{adjustbox}


% Some shortcuts
\newcommand{\cov}{\mathrm{cov}}
\newcommand{\dif}{\mathrm{d}}
\newcommand{\bigbrk}{\vspace*{2in}}
\newcommand{\smallbrk}{\vspace*{.1in}}
\newcommand{\midbrk}{\vspace*{1in}}
\newcommand{\red}[1]{{\color{red}#1}}
\newcommand{\blue}[1]{{\color{blue}#1}}
\newcommand{\green}[1]{{\color{green}#1}}
\newcommand{\calc}[1]{{\fbox{\mbox{#1}}}}
\newcommand{\Var}{\mathrm{Var}}%
\newcommand{\Cov}{\mathrm{Cov}}%

\usetheme[numbering=fraction]{metropolis}
% section page?
% \usetheme[sectionpage=none]{metropolis}
% place progressbar under frametitle?
% \usetheme[progressbar = frametitle]{metropolis}

% Add logo?
% This force a progress bar
\makeatletter
\setlength{\metropolis@frametitle@padding}{1.8ex}% <- default 2.2 ex
\setbeamertemplate{frametitle}{
    \nointerlineskip
    \begin{beamercolorbox}[
        wd=\paperwidth,
        sep=0pt,
        leftskip=\metropolis@frametitle@padding,
        rightskip=\metropolis@frametitle@padding,
        ]{frametitle}
        \metropolis@frametitlestrut@start
        \insertframetitle
        \nolinebreak
        \metropolis@frametitlestrut@end
        \hfill
        \includegraphics[height=3ex, keepaspectratio, valign=c]{UTDmono_circle_flame.png}
    \end{beamercolorbox}
    \usebeamertemplate*{progress bar in head/foot}
}

% To center title?
\setbeamertemplate{title page}[default]
% To add a rectangle box for title?
% \setbeamercolor{titlelike}{parent = frametitle}   

% item color and shape
% \setbeamertemplate{itemize item}{\color{mDarkTeal}$\bullet$}
% \setbeamertemplate{itemize subitem}{\color{mDarkTeal}$\bullet$}
% \setbeamertemplate{itemize subsubitem}{\color{mDarkTeal}$\bullet$}

% UTD colors
\definecolor{warmgray10}{RGB}{118,106,98}
\definecolor{warmgray2}{RGB}{213,210,202}
\definecolor{warmgray1}{RGB}{230,230,230}
\definecolor{flameOrange}{RGB}{199,91,18}
\definecolor{ecoGreen}{RGB}{0,133,86}

% only change item color 
\setbeamercolor*{itemize item}{fg = mDarkTeal}
\setbeamercolor*{itemize subitem}{fg = mDarkTeal}
\setbeamercolor*{itemize subsubitem}{fg = mDarkTeal}
% enumerate color
\setbeamercolor*{enumerate item}{fg = mDarkTeal}
\setbeamercolor*{enumerate subitem}{fg = mDarkTeal}
\setbeamercolor*{enumerate subsubitem}{fg = mDarkTeal}
% description color
\setbeamercolor*{description item}{fg = mDarkTeal}
\setbeamercolor*{description subitem}{fg = mDarkTeal}
\setbeamercolor*{description subsubitem}{fg = mDarkTeal}

% define blocks
\newenvironment<>{problock}[1]{
  \begin{actionenv}#2
    \def\insertblocktitle{#1}
    \par
    \mode<presentation>{
      \setbeamercolor{block title}{fg = white, bg = ecoGreen} %mDarkTeal}
      \setbeamercolor{block body}{fg = black, bg = warmgray1}
    }
    \usebeamertemplate{block begin}}
  {\par\usebeamertemplate{block end}\end{actionenv}}

\newenvironment<>{defblock}[1]{
  \begin{actionenv}#2
    \def\insertblocktitle{#1}
    \par
    \mode<presentation>{
      \setbeamercolor{block title}{fg = white,bg = flameOrange} %mDarkBrown}
      \setbeamercolor{block body}{fg = black,bg = warmgray1}
    }
    \usebeamertemplate{block begin}}
  {\par\usebeamertemplate{block end}\end{actionenv}}

% \setbeamerfont{block title}{size={}}
% \setbeamercolor{titlelike}{parent=structure,bg=flameOrange!80!black,fg=white} % title
\mode
<all>

% fancy for Verbatim?
\fvset{frame = single, framesep = 1mm,fontfamily = courier,
  fontsize = \scriptsize, numbers = left, framerule = .3mm,
  numbersep = 1mm,commandchars = \\\{\}}

% reset text width
\setbeamersize{text margin left = 5pt,text margin right = 5pt}

\title[Short title]{Long title\\ Secondary title \\ 
  \small{Additional notes}}
\author[Author name]{Author name}
% \institute[UTD]{Department of Mathematical Sciences \\ The University of Texas at Dallas}
\date{}
% UTD logo on title page
% \titlegraphic{\includegraphics[width=5cm]{mono_nsm_print_header.jpg}}
\titlegraphic{\includegraphics[scale = .27]{UTDmono_NSM_header_RGB-1.png}}

\begin{document}

% default title thicker line
% need to adjust
% \frame{
% \maketitle
% \begin{tikzpicture}[remember picture, overlay]
%   \def\xmargin{.1cm}     % Left and right margin of the title line
%   \def\yshift{-0.1cm}   % Moves the title line upwards (-1.25cm moves title line downwards)
%   \draw[mLightBrown,line width=2pt] (current page.west) ++(\xmargin,\yshift) -- ++({\paperwidth-2*\xmargin},0);
% \end{tikzpicture}
% }


\let\otp\titlepage
\renewcommand{\titlepage}{\otp\addtocounter{framenumber}{-1}}

\frame[plain]{
  \vspace{1cm}
  \maketitle
  \begin{tikzpicture}[remember picture, overlay]
    \def\xmargin{.8cm}     % Left and right margin of the title line
    \def\yshift{0.6cm}   % Moves the title line upwards 
    \draw[mLightBrown,line width=2.2pt] (current page.west) ++(\xmargin,\yshift) -- ++({\paperwidth-2*\xmargin},0);
  \end{tikzpicture}
}


% Set up UTD backgroud
% Show background from here on
\setbeamercolor*{item}{fg=red}
\bgroup
\usebackgroundtemplate{
  \tikz[overlay,remember picture] \node[opacity=0.05, at=(current page.center)] {
    \includegraphics[height=\paperheight,width=\paperwidth]{UTDbg}};}


\begin{frame}{Table of contents}
  \setbeamertemplate{section in toc}[sections numbered]
  \tableofcontents[hideallsubsections]
\end{frame}

\section{Introduction}
\begin{frame}{metropolis theme}
  \begin{itemize}
  \item I created this template for presentation slides under the (unofficial) theme of the \alert{UTD}.
  \item This template uses the metropolis theme \url{https://github.com/matze/mtheme}.
  \item To install do, 
    \begin{enumerate}
    \item `git clone git@github.com:matze/mtheme.git'
    \item `make sty' inside of the folder
    \item Remove the folder mtheme
    \end{enumerate}
  \end{itemize}
\end{frame}

\begin{frame}{Required files}
  \begin{itemize}
  \item Here is the set of files to be used: 
    \begin{description}
    \item[slide.tex] main file for text
    \item[slide.bib] bib file for citations
    \item[UTDbg.jpg] background image 
    \item[mono\_nsm\_print\_header.jpg] title logo
    \end{description}
  \item Compile \textbf{slide.tex} twice to generate the pdf file of this slide
  \end{itemize}
\end{frame}


\section{Blocks}
\begin{frame}{Blocks}
  \begin{problock}{Exercise block}
    Some exercise
  \end{problock}
  \begin{defblock}{Definition block}
    Some definition
  \end{defblock}
\end{frame}

\section{Adding extras}
\begin{frame}[fragile]
  \frametitle{Adding codes}
  \begin{itemize}
  \item Print codes with \texttt{Verbatim}. 
    \begin{Verbatim}
> print("Hello world")
[1] "Hello world"
    \end{Verbatim}
  \item This works fine, but can be improved by migrating to \texttt{knitr} or \texttt{Rmarkdown}.
  \end{itemize}
\end{frame}

\begin{frame}{Table}
  \begin{itemize}
  \item A good online \LaTeX table generator \url{https://www.tablesgenerator.com/}
  \end{itemize}
  \begin{table}[h]
    \caption{April 2019}
    \begin{tabular}{rrrrrrr}
      \toprule
      Sun & Mon & Tue & Wed & Thu & Fri & Sat \\
      \midrule
          & 1 & 2 & 3 & 4 & 5 & 6 \\
      7 & 8 & 9 & 10 & 11 & 12 & 13\\
      14 & 15 & 16 & 17 & 18 & 19 & 20 \\
      21 & 22 & 23 & 24 & 25 & 26 & 27 \\
      28 & 29 & 30 \\
      \bottomrule
    \end{tabular}
  \end{table}
\end{frame}

\begin{frame}{Equation}
  \begin{itemize}
  \item A good online \LaTeX equation generator \url{https://www.codecogs.com/latex/eqneditor.php}
    \begin{equation*}
      \int_{-\infty}^\infty\frac{1}{\sqrt{2\pi\sigma^2}}e^{-\frac{(x - \mu)^2}{2\sigma^2}}=1
    \end{equation*}
  \end{itemize}
\end{frame}

\begin{frame}{bib}
  \begin{itemize}
  \item If a bib file is presence, be sure to run bibtex and pdflatex twice.
    \begin{description}
    \item[\citet{chiou2017rank}] proposed methods for length-biased data
    \item[\citet{chiou2018semiparametric}] proposed methods for panel count data
    \item[\citet{xu2020generalized}] proposed methods for recurrent event data
    \end{description}
  \item Citation can be mentioned passively.
  \item \textit{aftgee} is a useful R package for fitting survival data \citep{chiou2014fitting}.
  \end{itemize}
\end{frame}


\begin{frame}[allowframebreaks]%[shrink = 5]
  \begin{center}
    \scriptsize
    \bibliographystyle{abbrvnat}
    \bibliography{slides}
  \end{center}
\end{frame}

\end{document}